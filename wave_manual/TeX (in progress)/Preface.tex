\chapter{Preface}

The need for more careful management of our planet's land and water resources is quickening. This results from two clashing realities: the growing numbers of fellow humans with their increasing expectations for a more bountiful life, and the limitations of the planet's natural resources and ability to absorb environmental abuse.\\

Excessive use of agrochemicals in many industrialized countries endanger the quality of land and water resources and will do so if no drastic measures are taken. In Europe for example the drinking water quality standard of 50 ppm nitrates is exceeded at several locations. Legislation measures for controlling fertilizer, pesticide and herbicide applications are currently considered at regional, national and transnational levels. Several authorities have already enacted different measures to reduce nitrate pollution. Most of these measures are based on rough estimations of nitrogen leaching risk, which not always consider the interaction between climate, crop, soil and geo-hydrology. Even if these effects are considered, it is not clear what the long term effect of regulative measures will be. Both the developments of regulations and the assessment of their long term effects would be substantially simplified by the availability of comprehensive and tested simulation models. Such models can help the decision makers to define about when, where and how much fertilizer or pesticide to apply on agricultural land to ensure that the soil, surface and groundwater resource system are least effected.\\

The foregoing requires good understanding of the transfer and transformation processes of solutes as they migrate through the soil, and as such requires a good knowledge of the physical, chemical and biological laws governing the changes in the atmosphere-plant-soil continuum. Although considerable expertise has been gained in each of these domains, only recently due to the development of simulation models the interactions of the different processes effecting the fate of chemicals in the vadose environment are studied. As a result, a tremendous interest in system studies, using mathematical modelling, has emerged the last decade. Mathematical modelling is increasingly applied in environmental studies because it enables to get a better insight in processes, integrate knowledge of different disciplines, analyse complex problems in a holistic way, and predict short and long-term impacts of changes in climate and farming on the environment.\\

The major purpose of this reference and user's manual is to convey a modular modelling system for studying the transport and transformations of matter and energy in the soil, crop and vadose environment. The version of the WAVE model, presented herein, is based on the SWATRER (the extended and revised version of the Soil Water and Actual Transpiration Rate model of Feddes et al., 1978 (Dierckx et al., 1986)). After 1986, this version was regularly upgraded and extended within the frame of different doctoral research projects of the Institute for Land and Water Management of the K.U.Leuven. This resulted in several upgrades of the SWATRER programme and the development of models like SWATNIT (Vereecken et al., 1989). The latter is a programme for modelling the behaviour of nitrogen species in the soil-plant continuum. However, due to the different upgrading and adding of new sections the programme became rather complex and difficult to manipulate. Therefore, it was decided to completely restructure and rewrite the code of the programme to improve its transparency and user friendliness. The programme has been structured so that it will be easy to add new modules if knowledge of other soil processes then those already present in the programme, and relevant for the analysis of the fate and transfer of chemicals in soils, becomes available. The new programme code was given the name WAVE, which stands for simulation of the substances Water and Agrochemicals in the soil, crop and Vadose Environment.

To simplify the use of the reference and user manual, the chapters are presented in the sequence the WAVE-model describes the different processes. Furthermore, most of the analytical material is presented in a brief form with limited attention given to the derivation of the equations presented. This has been done to focus on the synthesis of the entire process rather than concentration on the analysis of the individual steps along the way. Throughout the different sections of the reference manual special emphasis is given to different parametric models for the description of model parameters. In addition, the manual contains extensive tables with published values of difficult to measure parameters and factors. In this way the manual serves also the objective of providing a comprehensive literature review of parametric models used to describe certain model parameters and value ranges for model parameters. Besides a description of the main transport modules (water, solutes and energy), the crop growth and the nitrogen fate module, the reference and user's manual devotes quite some attention to the model 'input' and 'output', and the description of the structure of the programme code and the variables used.\\

First and foremost, this is a reference and user's manual for both teachers, researchers and students. Its objective is to present and convey the state-of-the-art in the field of modelling some physical, chemical and biological processes in the vadose environment, which control the fate and transfer of agrochemicals. Therefore, the emphasis is on approaches for conceptualizing, applying, and synthesizing basic underlying concepts. For the convenience of engineers there is an extensive section on how to use the programme code, and in rewriting the code considerable attention is given in adding special check routines for controlling the consistency of the model input.\\

The reference and user's manual is written from the perspective of authors with scientific background in soil physics, agronomy and computer science and skill in the measurement of static and dynamic soil properties, and the monitoring of complex laboratory and field experiments. In addition, the authors are experienced in fields such as model calibration and validation, and in quantifying the effect of poor concept formulation and random input uncertainty on model output. Furthermore, they have gone through several exercises of using the WAVE-model for engineering purposes. The manual integrates the efforts of the Institute for Land and Water Management on grappling during 20 years of how best to understand and model some dominant processes in the vadose environment. Since continuous developments and research findings will become available, the authors are convinced that this reference and user's manual in future will be replaced by new updates.\\


Professor Jan FEYEN\\
Head of the Institute for Land and Water Management\\
Katholieke Universiteit Leuven