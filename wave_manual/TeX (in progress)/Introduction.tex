\chapter{Introduction}

\section{Background and ORIGIN}

Expansion of human activities causes dispersion of pollutants in the subsurface environment. Today, acid rain, hazardous chemical wastes, fertilisers, pesticides, heavy metals, nuclear deposits, solvents and sewage sludge are amongst other things, a serious threat to soil and groundwater quality. Therefore, existing groundwater and soil conservation policies and strategies are reconsidered, while new are developed. For the development of adequate and durable measures the system analytic approach offers many interesting features. A system approach, encompassing the development and validation of simulation models, can help decision makers and scientists to get better insights in the complexity and the interaction of the different processes affecting the fate of chemicals in the dynamic soil-crop environment. As a result, a tremendous interest in system studies, especially through mathematical modelling, has emerged the last decade. Mathematical modelling is an important part of many current environmental studies and it is believed that there is still a lot of scope for model development as long as new insights in processes will emerge and computing facilities improve.\\

The WAVE-model (Water and Agrochemicals in soil, crop and Vadose Environment), is an example of such a mathematical tool. The model describes the transport and transformations of matter and energy in the soil, crop and vadose environment. The model is mainly process-based, since physical, chemical and biological laws were considered when developing the model. The model is deterministic, by which is meant that one set of input data always yields the same model output values. The model is numerical, since finite difference techniques were used for the solution of the differential equations describing matter and energy transport in the soil-crop continuum. The model is holistic, which means that an attempt was made to integrate the different sub-processes (and hence sub-models) ruling the transfer and fate of different state variables in the complex soil-crop environment. The model is one-dimensional, because it is assumed that governing transport processes of matter and energy in the soil subsystem occur essentially in the vertical direction. The model is an explanatory model because it helps to understand the different processes and process interactions governing e.g. pollutants in the soil. However, results form these explanatory studies can always be used in extrapolation or prediction studies for decision making. Hence, the current model is one of the ad-hoc tools available to improve current management of the soil-crop environment. It is a unique tool for the better understanding of the processes controlling the transfer and fate of chemicals in soils, the evaluation of experimental field data, the prediction of short and long-term impact of farming strategies on the quality of soil and the groundwater and the development of soil specific environmental measures for the application of fertilisers.\\

The present WAVE-model is the output of different research projects which aimed at the development, calibration and validation of mathematical models for the quantitative description of matter and energy in soils and environment. These projects were funded by the Institute for the Encouragement of Scientific Research in Agriculture and Industry, Belgium (I.W.O.N.L.) and the Directorates General VI and XII of the European Union. The projects output is, amongst other things, a model consisting of different modules simulating respectively the fate of water, solute, heat, or nitrogen in the soil environment and crop growth. The model is structured in a modular way, enabling the user to use only those modules required to analyse his problem. This allows also the extension of the present model with other modules without the need to adapt the model structure or existing input files of the model. It offers the possibility to exchange modules when new concepts and insights of certain processes becomes available. Fig. 1 present the different modules and the arrows indicate the 'uses-relationships' among them. For example, the solution of the solute flow equation needs to be proceeded by the solution of the water flow equation. Hence the solute module 'uses' the water flow module, which is indicated by the direction of the arrow.\\


Fig. 1: Schematic presentation of the modules in WAVE (release 1.0). Full line arrows represents obligatory 'uses-relations', dashed lines are optional\\

The WAVE-model is a software package developed by the Institute for Land and Water Management of the K.U.Leuven, Belgium. The present version of the model integrates earlier models and packages developed by the Institute or developed and published by other scientific institutes. The model is a revised version of the SWATNIT-model (Vereecken et al., 1990; 1991), which integrates the SWATRER-model (Feddes et al., 1978; Belmans et al., 1983; Dierckx et al., 1986) , a nitrogen model based on the SOILN-model (Bergstrom et al., 1991), a heat and solute transport model based on the LEACHN-model (Wagenet and Hutson, 1989) and the universal crop growth model SUCROS (van Keulen et al., 1982; Spitters et al., 1988). New modules, considering macroporous flow, pesticide flow, amongst other things are in current development, and will be added after extensive validation. The WAVE-model is written in MS-FORTRAN 5.10 and can be run either under UNIX or MS-DOS.\\

The user's manual is a revised form of the report of Vanclooster et al. (1993). In the user's manual an attempt is made to describe in detail all the process formulations. Special emphasis is given to the quotation of recent and relevant literature and to the listing of default values for model parameters and constants. This information together with the integration of automatic data quality control checks in the input files and the use of menus will certainly facilitate the use of the WAVE-model.\\

\section{Space and time scales}

WAVE is essentially a 1-D model for the description of matter and energy flow in the soil and crop system. Mass and energy fluxes in the soil system are known to be strong non-linear processes. The numerical solution of the 3-D transport problem for unsteady state boundary conditions is, from a computational point of view, still an arduous task. Hence the model is conceived to describe flow only in 1-D systems, as in soil laboratory columns or field lysimeters. The model can also be used to describe transport at the field scale (or a small pedon) if transport is mainly vertical and if effective (1-D) parameters are used. If not, the model describes only flow for a horizontally isolated pedestals (Fig. 2).\\


Fig. 2: Concept of the horizontal space scale in the WAVE-model\\

In the vertical direction, the model considers the existence of heterogeneity in the form of soil layers within a soil profile (Fig. 3). The soil layers are subdivided in space intervals called the soil compartments. Halfway each soil compartment a node is identified, for which state variable values are calculated using finite difference techniques. All soil compartments have the same thickness and the user can specify the thikness depending on the desired acuracy. Increasing the compartment thickness will decrease the calculation time but also the numerical accuracy.\\

Fig. 3: Concept of the vertical space scale\\

The WAVE-model uses a time step smaller then a day to calculate the different system state variables, for processes which are strongly dynamic (water transport, heat transport, solute transport, solute transformations). The time step is variable, and is chosen as to limit mass balance errors induced by solving the water flow equation. However, the time step size criterion can be input to change the model's robustness. For less dynamic processes (crop growth) a fixed daily time step is often used. The model input is specified on a daily basis and flux type boundary conditions, are assumed constant within the time span of a day. This means, for example, that the daily precipitation is distributed equally within the day. State variables are integrated after each day to yield daily output. The simulation period should not exceed one single year. The simulation starts at midnight of the specified starting date and ends at midnight of the specified ending date.\\

\section{References}

Belmans, C., J.G. Wesseling and R.A. Feddes, 1983. Simulation of the water balance of a cropped soil: SWATRE. J. of Hydrol., 63:271-286.

Bergstrom, L., H. Johnsson and G. Tortensson, 1991. Simulation of nitrogen dynamics using the SOILN model. Fert. Res., 27:181-188.

Dierckx, J., C. Belmans and P. Pauwels, 1986. SWATRER, a computer package for modelling the field water balance. Reference manual. Soil and Water Engng. Lab., K.U.Leuven, Belgium, 114 pp.

Feddes, R.A. , P.J. Kowalik and H. Zaradny, 1978. Simulation of field water use and crop yield. Simulation Monographs, PUDOC, Wageningen., The Netherlands. 189 pp.

Spitters, C.J.T, H. Van Keulen and D.W.G. Van Kraailingen, 1988. A simple but universal crop grwoth simulation model, SUCROS87. In: R. Rabbinge, H. Van Laar and S. Ward (eds.). Simulation and systems management in crop protection. Simulation Monographs, PUDOC, Wageningen, The Netherlands.

Van Keulen, H., F.W.T. Penning de Vries and E.M. Drees, 1982. A summary model for crop growth. In: F.W.T. Penning de Vries and H.H. van Laar (eds.) Simulation of crop growth and crop production. PUDOC, Wageningen, 87-98.

Vanclooster, M., J. Diels, P. Viaene and J. Feyen, 1993. The SWAT-modules: Theory and input requirement. Internal Report 10. Institute for Land and Water Management, K.U.Leuven, 86 pp.

Vereecken, H., M. Vanclooster and M. Swerts, 1990. A simulation model for the estimation of nitrogen leaching with regional applicability. In: R. Merckx and H. Vereecken (eds.). Fertilization and the environment. Leuven Academic Press, Belgium, 250-263.

Vereecken, H., M. Vanclooster, M. Swerts and J. Diels, 1991. Simulating water and nitrogen behavior in soil cropped with winter wheat. Fert. Res. , 27: 233-243.

Wagenet, R.J.and J.Hutson, 1989. LEACHN, a process-based model of water and solute movement, transformations, plant uptake and chemical reactions in the unsaturated zone. Centre for Environ. Res., Cornell University, Ithaca, NY, 147 pp.
